Условие задачи: Рассмотрим задачу о пьяном прохожем, которую называют еще задачей о случайном блуждании. Предположим, что пьяный, стоя на углу улицы, решает прогуляться. Пусть вероятности того, что, достигнув очередного перекрестка, он пойдет на север, юг, восток или запад, одинаковы. Оценить вероятность того, что, пройдя 10 кварталов, пьяный окажется не далее двух кварталов от места, где он начал прогулку. Оценку вероятности провести на основании 1000 испытаний.

Решение задачи (листинг):

\begin{minted}{python}
import numpy as np
from scipy.integrate import solve_ivp
import matplotlib.pyplot as plt

# Параметры задачи
steps = 10  # Количество шагов
radius = 2  # Радиус от начальной точки
trials = 1000  # Количество испытаний

# Функция для моделирования случайного блуждания
def random_walk(steps):
    """
    Симулирует случайное блуждание.
    """
    # Направления: север (0,1), юг (0,-1), восток (1,0), запад (-1,0)
    directions = [(0, 1), (0, -1), (1, 0), (-1, 0)]
    position = np.array([0, 0])
    for _ in range(steps):
        step = directions[np.random.choice(4)]
        position += step
    return position

# Моделирование и подсчет успешных случаев
success_count = 0
for _ in range(trials):
    final_position = random_walk(steps)
    distance = np.linalg.norm(final_position, ord=2)  # Евклидово расстояние
    if distance <= radius:
        success_count += 1

# Оценка вероятности
probability = success_count / trials

# Вывод результата
print(f"Вероятность того, что пьяный окажется не далее двух кварталов: {probability:.4f}")
\end{minted}

Вывод:
\begin{minted}{shell}
    Вероятность того, что пьяный окажется не далее двух кварталов: 0.4090
\end{minted}