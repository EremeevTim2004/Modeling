Условие задачи: Случайные величины $X$ и $Y$ независимы и имеют нормальное распределение с математическим ожиданием $0$ и дисперсией $1$. Оценить $P(|X-Y|<=1)$ на основании $1000$ выборочных значений пар $(x_{i}, y_{i})$, $i=1,...,1000$.

Решение задачи (листинг):

\begin{minted}{python}
import numpy as np
from scipy.integrate import solve_ivp
import matplotlib.pyplot as plt

# Параметры нормального распределения
mean = 0
std_dev = 1

# Количество выборочных значений
n_samples = 1000

# Генерация выборок для X и Y
X = np.random.normal(mean, std_dev, n_samples)
Y = np.random.normal(mean, std_dev, n_samples)

# Вычисление |X - Y| и подсчет случаев, удовлетворяющих условию
absolute_differences = np.abs(X - Y)
success_count = np.sum(absolute_differences <= 1)

# Оценка вероятности
probability_estimate = success_count / n_samples

# Вывод результата
print(f"Оценка вероятности P(|X - Y| <= 1): {probability_estimate:.4f}")
\end{minted}

Вывод:
