Условие задачи: Пусть в течение суток длительность интервала времени между очередными поступлениями трамваев на остановку распределена экспоненциально с математическим ожиданием равным 15 минутам. Один раз в сутки пассажир приходит на остановку в случайный момент времени суток (равномерное распределение длительности) и ожидает очередной трамвай в течение времени $\Delta$. Провести 1000 испытаний с моделью и оценить математическое ожидание $\Delta$.

Решение задачи (листинг):

\begin{minted}{python}
import numpy as np

# Параметры задачи
mean_interval = 15  # Средний интервал между трамваями (в минутах)
rate = 1 / mean_interval  # Параметр экспоненциального распределения
n_trials = 1000  # Количество испытаний

# Симуляция
waiting_times = []
for _ in range(n_trials):
    # Момент прихода пассажира (в минутах с начала суток)
    passenger_arrival = np.random.uniform(0, 1440)  # 1440 минут в сутках
    # Генерация интервалов между трамваями в течение суток
    tram_arrivals = [0]
    while tram_arrivals[-1] < 1440:
        tram_arrivals.append(tram_arrivals[-1] + np.random.exponential(scale=mean_interval))
    # Удаляем последний элемент, если он больше 1440 минут
    tram_arrivals = np.array(tram_arrivals[:-1])
    # Проверка на наличие следующего трамвая
    next_trams = tram_arrivals[tram_arrivals >= passenger_arrival]
    if next_trams.size == 0:  # Если массив пуст, возвращаемся к первому трамваю следующего дня
        waiting_time = 1440 - passenger_arrival + tram_arrivals[0]
    else:
        waiting_time = next_trams.min() - passenger_arrival
    waiting_times.append(waiting_time)

# Оценка математического ожидания времени ожидания
expected_waiting_time = np.mean(waiting_times)
# Вывод результата
print(f"Оценка математического ожидания времени ожидания (\u0394): {expected_waiting_time:.2f} минут")
\end{minted}

Вывод:
\begin{minted}{shell}
    ценка математического ожидания времени ожидания (Δ): 14.99 минут
\end{minted}